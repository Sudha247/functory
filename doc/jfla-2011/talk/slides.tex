\documentclass{beamer}
%\documentclass[handout]{beamer}

\usepackage[latin1]{inputenc}
\usepackage[french]{babel}

\definecolor{kwblue}{rgb}{0.67,0.12,0.92}
\definecolor{ceruleanblue}{rgb}{0, 0.48, 0.65}
\definecolor{lightpink}{rgb}{1., 0.71, 0.75}
\definecolor{lightblue}{rgb}{0.8,0.8,1}
\definecolor{lightred}{rgb}{1,0.8,0.8}

\let\emph\alert

\begin{document}

\title{Functory : Une biblioth�que de calcul distribu� \\ pour
  Objective Caml}
\author[Jean-Christophe]{Jean-Christophe Filli\^atre \& Kalyan Krishnamani}
\date{JFLA, 31 janvier 2011}

\begin{frame}
  \titlepage
  \pgfimage[height=8mm]{cnrs-logo2}\hfill
  \pgfimage[height=6mm]{saclay}\hfill
  \pgfimage[height=10mm]{lrilogo}\hfill
  \pgfimage[height=8mm]{upsudlogo}
\end{frame}

\begin{frame}\frametitle{}
  motivating example = SMT solvers on numerous VC during the night, 3
  machines, etc.
\end{frame}

\begin{frame}\frametitle{}
  requirements
  \begin{itemize}
  \item ocaml library
  \item fault tolerance
  \item user-friendly API
  \end{itemize}
\end{frame}

\begin{frame}\frametitle{}
  design a la Map/Reduce (1/2 slides)

  TODO : a nice picture here
\end{frame}

\begin{frame}\frametitle{}
  API main idea = compute function
\end{frame}

\begin{frame}\frametitle{}
  simpler situations $=>$ simpler implementations but same API

  sequential
  cores
\end{frame}

\begin{frame}\frametitle{}
  network and marshaling considerations

  if both programs are the same, API is unchanged
\end{frame}

\begin{frame}\frametitle{}
  but sometimes you can't afford running the same program as master
  and workers $=>$ not more marshaling of closures $=>$ slightly different API
\end{frame}

\begin{frame}\frametitle{}
  derived API

  (mention that details are in the paper)
\end{frame}

\begin{frame}\frametitle{}
  implementation details

  first, cores, reusing the picture to explain that workers are simply
  forks (no control over scheduling)

  explain the main loop with pending tasks and associated data structure
\end{frame}

\begin{frame}\frametitle{}
  network implementation

  first, communication model (client/server, TCP/IP, etc.), reusing
  the picture 
\end{frame}

\begin{frame}\frametitle{}
  fault tolerance, protocol

  \begin{center}
    \includegraphics{state.mps}
  \end{center}
\end{frame}

\begin{frame}\frametitle{}
  benchmarks

  SMT solvers first
\end{frame}

\begin{frame}\frametitle{}
  n-queens
\end{frame}

\begin{frame}\frametitle{}
  Mandelbrot
\end{frame}

\begin{frame}\frametitle{}
  matrix multiplication
\end{frame}

\begin{frame}\frametitle{}
  conclusion, future work, etc.
\end{frame}


\end{document}

%%% Local Variables: 
%%% mode: latex
%%% TeX-master: t
%%% End: 

