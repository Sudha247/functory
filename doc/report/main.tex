%-----------------------------------------------------------------------------
%
%               Template for sigplanconf LaTeX Class
%
% Name:         sigplanconf-template.tex
%
% Purpose:      A template for sigplanconf.cls, which is a LaTeX 2e class
%               file for SIGPLAN conference proceedings.
%
% Author:       Paul C. Anagnostopoulos
%               Windfall Software
%               978 371-2316
%               paul@windfall.com
%
% Created:      15 February 2005
%
%-----------------------------------------------------------------------------


\documentclass[a4paper,12pt]{article}

% The following \documentclass options may be useful:
%
% 10pt          To set in 10-point type instead of 9-point.
% 11pt          To set in 11-point type instead of 9-point.
% authoryear    To obtain author/year citation style instead of numeric.

\usepackage{fullpage,comment,amsmath,graphicx,color,url}

\newcommand{\Ocaml}{OCaml}
\newcommand{\functory}{\textsf{Functory}}
\newcommand{\JoCaml}{Jo{\&\!}Caml}
\newcommand{\unix}{\textsc{Unix}}

\begin{document}

\title{Functory: Distributed Computing for the Common Man\thanks{This
    research was partly supported by the French national project U3CAT
    (\emph{Unification of Critical C Code Analysis Techniques},
    ANR-08-SEGI-021).}} 
%\subtitle{Subtitle Text, if any}

\author{Jean-Christophe Filli\^{a}tre \quad K. Kalyanasundaram \\
  CNRS, LRI, Univ Paris-Sud 11, Orsay F-91405\\
  INRIA Saclay - \^{I}le-de-France, ProVal, Orsay, F-91893 \\
   \url{filliatr@lri.fr, kalyan.krishnamani@inria.fr}}

\maketitle

\begin{abstract}
  We present a distributed computing library for
  functional programming languages. The main features of this library
  include (1) a generic API, (2) several implementations to
  adapt to different computing environments such as sequential,
  multi-core or network, and (3) a reliable fault-tolerance mechanism.
  This paper describes the motivation behind this work, as well as
  the design and implementation of the library. It also demonstrates
  the potential of the library using realistic experiments.
\end{abstract}

%\keywords

\section{Introduction}

This paper introduces a generic library for distributed computing. This
work was initially motivated by the computing needs that exist in our
own research team. Our applications include large-scale deductive
program verification, which amounts to checking the validity of a
large number of logical formulas using a variety of automated theorem
provers~\cite{filliatre07cav}. Our computing infrastructure consists
of a few powerful multi-core machines (typically 8 to 16 cores) and
several desktop PCs (typically dual-core). However, for our
application needs, there is no library that helps in exploiting such a
computing infrastructure in our favorite functional programming
language.  Hence we designed and implemented such a
library\footnote{This library is implemented in \Ocaml, but the
  implementation should be straightforward in any functional
  programming language.}, which is the subject of this paper.

The distributed computing library presented in this paper is not a
library that helps in parallelizing computations. Rather, it provides
facilities for reliable, distributed execution of parallelizable
computations. In particular, it provides a set of APIs that allows the
execution of large-scale parallelizable computations, very relevant to
our application needs (and presumably relevant to a variety of
real-world applications), over multiple cores in the same machine or
over a network of machines. 
The most important features of our library are the following:
\begin{itemize}
\item \emph{Genericity}: 
  it allows various patterns of polymorphic computations;
\item \emph{Simplicity}: switching between multiple cores on the same
  machine and a network of machines is as simple as changing a couple
  of lines of code;
\item \emph{Task distribution and fault-tolerance}: 
  it relieves the user from implementing task distribution routines
  and reliable fault-tolerance mechanisms.
\end{itemize}
It is worth noting that the library is not targeted at applications
running on server farms, crunching enormous amounts of data. Neither
is it limited to research labs possessing few powerful
machines. It serves a variety of users and a wide
spectrum of needs, from desktop PCs to networks of machines, and hence
the title.

The application domain of such a distributed computing library is
manyfold.  The most striking application, as also relevant to our
research endeavours, is in validating thousands of verification
conditions using automated theorem provers. Having noted this, it
should be apparent that the library finds use in efficient execution
of any large-scale expensive computation that is parallelizable.
% what it is and what it is not
% \begin{itemize}
% \item it is not a library which does parallelization
% \item it is a library for reliable distributed execution of
%   parallelizable computation 
% \end{itemize}
% inspired by Google's MapReduce~\cite{mapreduce} (itself inspired by functional
% programming, ironically) ;  however, there are differences:
% \begin{itemize}
% \item polymorphic, higher-order API
% \item more generic (cores/network)
% \item does not focus on association lists
% \item we don't have data locality (future work)
% \end{itemize}
% what is the target audience/applications: 
% for instance, use if automatic provers on thousands of verification
% conditions, on computing infrastructures which can be
% \begin{itemize}
% \item a single machine with multiple cores, possibly remote
% \item several machines, small or large in number, over a network
% \end{itemize}
\paragraph{Distributed Computing.}
A typical distributed computing library provides the following (we
borrow some terminology from Google's MapReduce~\cite{mapreduce}):
\begin{itemize}
\item A notion of \emph{tasks} which denote atomic computations
  to be performed in a distributed manner; 
\item A set of processes (possibly executing on remote machines)
  called \emph{workers} that perform
  the tasks, producing results;
\item A single process called a \emph{master} which is in charge
  of distributing the tasks among the workers and managing results
  produced by the workers.
\end{itemize}
In addition to the above, distributed computing environments also
implement mechanisms for fault tolerance, efficient storage, and
distribution of tasks. This is required to handle network failures
that may occur, as well as to optimize the usage of machines in the
network. Another concern of importance is the transmission of messages
over the network. This requires efficient 
\emph{marshaling} of data, that is encoding and decoding of data 
for transmission over different computing environments.  It is desirable to
maintain architecture independence while transmitting marshalled data,
as machines in a distributed computing environment often run on
different hardware architectures and make use of varying software
platforms. For example, machine word size or endianness may be different
across machines on the network.

\paragraph{A Functional Programming Approach.}
Our work was initially inspired by Google's
MapReduce\footnote{Ironically, Google's approach itself was inspired
  by functional programming primitives.}. However, our functional
programming environment allows us to be more generic. 
The main idea behind our approach is that
workers may implement any polymorphic function:
\begin{ocaml}
  worker: 'a -> 'b
\end{ocaml}
where \of{'a} denotes the type of tasks and \of{'b} the type of results.
Then the master is given a list of initial tasks, together with a
function to handle the results:
\begin{ocaml}
  master: ('a -> 'b -> 'a list) -> 'a list -> unit
\end{ocaml}
The function passed to the master is applied whenever a result is
available and may in turn generate new tasks (hence the return type
\of{'a list}).  The master is executed as long as there are pending
tasks.

Our library makes use of \Ocaml's marshaling capabilities as much as
possible. Whenever master and worker executables are exactly the same,
we can marshal polymorphic values and closures. However, it is not
always possible to have master and workers running the same
executable. In this case, we cannot marshal closures anymore but we
can still marshal polymorphic values as long as the same version of
\Ocaml\ is used to compile master and workers. When different versions
of \Ocaml\ are used, we can no longer marshal values but we can still
transmit strings between master and workers. Our library adapts to all
these situations, by providing several APIs.

\medskip
The paper is organized as follows. In Section~\ref{sec:API}, we
describe the main idea behind the generic API, and introduce
\of{master} and \of{worker} programs. 
% Section~\ref{sec:example} illustrates the API usage with an example.
In Section~\ref{sec:derived}, we
derive some interesting functions from the generic
API. Section~\ref{sec:implem} delves into the implementation details
of our library, while Section~\ref{sec:experiments} illustrates the
potential of the presented library through experimental evaluation. We
compare our approach with relevant related work in
Section~\ref{sec:future} and outline our future work.

\section{API}\label{sec:API}
In this section, we describe the
generic API which we shall extend with several specialized
functions in later sections. The main function in our API follows the idea
sketched in the introduction. It has the following signature:
\begin{ocaml}
  val compute : 
    worker:('a -> 'b) -> 
    master:('a * 'c -> 'b -> ('a * 'c) list) -> 
    ('a * 'c) list -> unit
\end{ocaml}
Tasks are pairs, of type \of{'a * 'c}, where the first component is
passed to the worker and the second component is local to the master.
The \ocaml{worker} function should be pure and is executed in parallel
in all worker processes. The function \ocaml{master}, on the
contrary, can be impure and is only executed in the master process.
The \ocaml{master} function typically stores results 
in some internal data structure.
The next section will describe the usage of this generic interface to
perform the traditional map and fold operations.

Actually, our library provides not just a single \ocaml{compute}
function as above, but instead five different versions
depending on the execution context. 
The five possible contexts are the following: 
\begin{enumerate}
\item \textbf{Purely sequential execution:}
  this is mostly intended for debugging;

\item \textbf{Several cores on the same machine:} 
  this implementation makes use of \unix\ processes and 
  provides a polymorphic \ocaml{compute} function;

\item \textbf{Same executable run on master and worker machines:}
  this implementation makes use of the ability to marshal \Ocaml\
  closures and  provides a polymorphic \ocaml{compute}
  function.
  Depending on whether the program is run as a master or as a worker,
  the relevant arguments of \ocaml{compute} are used.

\item \textbf{Master and workers are different programs, compiled with
    the same version of \Ocaml:} 
  we can no longer marshal closures but we can still
  marshal polymorphic values. As a consequence, 
  the \ocaml{compute} function is split into two
  polymorphic functions, to implement the master and workers 
  respectively:%
\vspace{-1em}
\begin{ocaml}
val Worker.compute : ('a -> 'b) -> unit -> unit
val Master.compute : 
  ('a * 'c -> 'b -> ('a * 'c) list) -> 
  ('a * 'c) list -> unit
\end{ocaml}
\item \textbf{Master and workers are different programs, not even
    compiled with the same version of \Ocaml:} we can no
  longer use marshaling, so the
  \ocaml{compute} function is split into two monomorphic functions
  over strings:%
\vspace{-1em}
\begin{ocaml}
val Worker.compute : (string -> string) -> unit -> unit
val Master.compute : 
  (string * 'c -> string -> (string * 'c) list) -> 
  (string * 'c) list -> unit
\end{ocaml}
  Note that the second component of each task is still polymorphic (of
  type \ocaml{'c} here), since it is local to the master.
\end{enumerate}

Our library is organized into three modules: \of{Sequential} for the
pure sequential implementation, \of{Cores} for multiple cores on the
same machine and \of{Network} for a network of machines, respectively.
The \of{Network} module itself is organized into three sub-modules,
called \of{Same}, \of{Poly} and \of{Mono},
corresponding to contexts 3, 4 and 5 above. 
The next section demonstrates the use of these modules.

% The different
% API functions are summarized in the following table (their signatures
% being given above):
% \begin{center}
%   \begin{tabular}{|c|c|c|c|c|}
%     \hline
%     \ocaml{Sequential} & \ocaml{Cores} &
%     \multicolumn{3}{|c|}{\ocaml{Network}} 
%     \\\cline{3-5}
%     &       & \ocaml{Same} & \ocaml{Poly} & \ocaml{Mono} \\\hline
%     \ocaml{compute}  & \ocaml{compute}   & \ocaml{compute}  &
%     \ocaml{master}  & \ocaml{master}  \\
%     & & & \ocaml{worker}  & \ocaml{worker} \\\hline
%   \end{tabular}
% \end{center}
% The following section introduces other interesting APIs that we
% derive from this generic API.

\begin{comment}
\section{Illustrative Example}\label{sec:example}

Let us assume we need to compute a sum such as
\begin{displaymath}
  s(n) = \sum_{i=0}^{n}f(i)
\end{displaymath}
where $f$ is a pure and computationally expensive function.
We start with the sequential implementation provided by the library:
\begin{ocaml}
  open Sequential
\end{ocaml}
The idea is to test our code on small values of $n$ before
parallelizing the computation for larger values.
Let us assume that $f$ is implemented as a top-level \of{worker} function:
\begin{ocaml}
  let worker i = ...
\end{ocaml}
and that $n$ is obtained from the command line:
\begin{ocaml}
  let n = int_of_string Sys.argv.(1)
\end{ocaml}
The list of tasks is simply the list of integers $0,1,\dots,n$,
computed as follows:
\begin{ocaml}
  let tasks = 
    let l = ref [] in 
    for i = 0 to n do l := (i, ()) :: !l done; 
    !l
\end{ocaml}
Each task is actually a pair, where the second component contains
some information which is kept local to the master process.
It is irrelevant here and thus the value \of{()} of type \of{unit} is used.
The current sum is kept into a global reference \of{s}:
\begin{ocaml}
  let s = ref 0
\end{ocaml}
The master function receives a task \of{(i,())} together with its
result \of{fi} and simply adds \of{fi} to \of{s}. There is no new task
generated, hence the returned empty list.
\begin{ocaml}
  let master (i,()) fi = s := !s + fi; []
\end{ocaml}
Finally, the whole computation takes place when we call the
\of{compute} function:
\begin{ocaml}
  let () = compute ~worker ~master tasks
\end{ocaml}

Now, let us assume we want to make use of a 4-core machine to compute
$s(n)$. It is as easy as just replacing the very first line with
\begin{ocaml}
  open Cores
  let () = set_number_of_cores 4
\end{ocaml}
while the rest of the code remains unchanged.

Later, we may want to use a network of machines instead, for example
two machines called \texttt{orcus} and \texttt{belzebuth} with 4 and 8
cores respectively. Similar to the above, we just replace the first
lines with
\begin{ocaml}
  open Network
  let () = declare_workers ~n:4 "orcus"
  let () = declare_workers ~n:8 "belzebuth"
  open Same
\end{ocaml}
Here, \of{Same} is a module which is to be used when master and worker
are running the same executable. It provides a \of{compute} function
which has the same signature as in modules \of{Sequential} and
\of{Cores}, so that the rest of the code remains unchanged. Master and
worker processes are distinguished at run-time using an environment
variable \texttt{WORKER} which is set/unset.

If we need to write two different programs for master and worker, for
reason of binary incompatibility or any other reason, the library API
is providing functions to do that. If master and worker are still
compiled with the same version of \Ocaml, we use the \of{Poly} module
which provides a polymorphic API. Let us start with the worker program.
It is as simple as
\begin{ocaml}
  open Poly
  let worker i = ...
  let () = Worker.compute worker ()
\end{ocaml}
The \of{Worker.compute} function enters a loop which waits for
tasks sent by the master and returns results computed using \of{worker}.
The master program is almost the same as before. First, we replace
module \of{Same} with module \of{Poly}:
\begin{ocaml}
  open Network
  let () = declare_workers ~n:4 "orcus"
  let () = declare_workers ~n:8 "belzebuth"
  open Poly
\end{ocaml}
Tasks and \of{master} function are unchanged:
\begin{ocaml}
  let tasks = ...
  let s = ref 0
  let master (i,()) fi = ...
\end{ocaml}
Finally, we start the computation with \of{Master.compute}, which does
not have a \of{worker} parameter anymore:
\begin{ocaml}
  let () = Master.compute ~master tasks
\end{ocaml}

When master and worker programs are compiled with different
versions of the \Ocaml\ compiler, our library still provides
a monomorphic API over strings. As a consequence, we need to convert
integers to and from strings in both master and worker.
The modified worker program is as follows:
\begin{ocaml}
  open Mono
  let worker i = ...
  let worker_string i = string_of_int (worker (int_of_string i))
  let () = Worker.compute worker_string ()
\end{ocaml}
The master program is modified in a similar way.
We simply replace \of{Poly} with \of{Mono} and encode/decode integers
as strings, as follows:
\begin{ocaml}
  let tasks = 
    let l = ref [] in 
    for i = 0 to n do l := (string_of_int i, ()) :: !l done; 
    !l
  let master (i,()) fi = s := !s + int_of_string fi; []
\end{ocaml}
\end{comment}

\section{Derived API}\label{sec:derived}

In most cases, the easiest way to parallelize an execution it to make
use of operations over lists, where processing of the
list elements are done in parallel. 
The previous section illustrated one such case.
To facilitate such a processing,
we now derive the most commonly used list operations from our generic
API.

The most obvious operation is the traditional map operation over
lists, that is:
\begin{ocaml}
  val map : ('a -> 'b) -> 'a list -> 'b list
\end{ocaml}
The next natural operation is a combination of map and fold
operations, that is a function like
\begin{ocaml}
 val map_fold :
   map:('a -> 'b) -> fold:('c -> 'b -> 'c) -> 
   'c -> 'a list -> 'c
\end{ocaml}
which, given two functions, an accumulator \of{a} and a list \of{l}, computes
\begin{equation}\label{eq:map-fold}
  \of{fold} ... (\of{fold} (\of{fold} ~ a ~ (\of{map} ~ x_1)) (\of{map} ~ x_2))
  ... (\of{map} ~ x_n)
\end{equation}
for some permutation $[x_1,x_2,...,x_n]$ of the list \of{l}.
We assume that the \of{map} operations are always performed in parallel.
Regarding \of{fold} operations, we distinguish two cases:
\begin{itemize}
\item either \of{fold} operations are computationally less expensive
  than \of{map} and we perform them locally;
\item or \of{fold} operations are computationally expensive and we
  perform them in parallel.
\end{itemize}
We thus provide two functions \of{map_local_fold} and \of{map_remote_}
\of{fold}.

In the case of \of{map_remote_fold}, only one \of{fold} operation can
be performed at a time (possibly in parallel with \of{map}
operations), as obvious from~(\ref{eq:map-fold}). However, there are
cases where several \of{fold} operations can be performed in parallel,
as early as intermediate results of \of{fold} operations are available.
This is the case when \of{fold} is an associative operation (which
implies that type \of{'b} and \of{'c} are the same).
Whenever
\of{fold} is also commutative, we can perform even more \of{fold}
operations in parallel. Thus our API provides two functions
\of{map_fold_a} and \of{map_fold_ac} for these two particular cases.
The five operations of the derived API are summarized
in Figure~\ref{fig:derived}.
\begin{figure}[t]
  \begin{ocaml}
    val map : 
      f:('a -> 'b) -> 'a list -> 'b list 
    val map_local_fold : 
      map:('a -> 'b) -> fold:('c -> 'b -> 'c) -> 
      'c -> 'a list -> 'c 
    val map_remote_fold : 
      map:('a -> 'b) -> fold:('c -> 'b -> 'c) -> 
      'c -> 'a list -> 'c 
    val map_fold_ac : 
      map:('a -> 'b) -> fold:('b -> 'b -> 'b) -> 
      'b -> 'a list -> 'b 
    val map_fold_a : 
      map:('a -> 'b) -> fold:('b -> 'b -> 'b) -> 
      'b -> 'a list -> 'b
  \end{ocaml}
  \caption{Derived API.}
\label{fig:derived}
\end{figure}

These five functions are actually derived from the generic API.  The
generality is achieved by implementing these derivations as a functor.
This functor, \of{Derive}, is parameterized by the \of{compute}
function, as follows:
\begin{ocaml}
module Derive
  (X : sig
     val compute : 
       worker:('a -> 'b) -> 
       master:('a * 'c -> 'b -> ('a * 'c) list) ->
       ('a * 'c) list -> unit
   end) = struct ... end
\end{ocaml}
We now explain how each function from Figure~\ref{fig:derived} is
implemented using \of{compute} inside the functor body.

\paragraph{\of{map}.} 
Function \of{map} is easily implemented by associating an integer
index with each element (of type \of{'a}) of the input list. Thus the
list of tasks passed to \of{compute} is a list of pairs of type \of{'a * int}.
As soon as an intermediate result is available, it is passed to
\of{master}, which stores it in a local table using the index as a key
and generates no new task. Whenever \of{compute} returns, we easily
recover the output list from the values stored in the table.

\paragraph{\of{map_local_fold}.} 
Function \of{map_local_fold} is implemented using a local reference
storing the current accumulator (of type \of{'c}).  As soon as an
intermediate result is available, the \of{master} combines it with the
accumulator using the function \of{fold}. Whenever \of{compute}
returns, we simply return the current accumulator.

\paragraph{\of{map_remote_fold}.}
In this case, combining the accumulator with an intermediate result is
itself a task to be performed in parallel with \of{map} operations.
Thus tasks are of two different kinds and we use a sum type to make
the distinction.
When an intermediate \of{map} result is available, the
accumulator may be already in use in a \of{fold} operation. Hence we
need to store the intermediate \of{map} result in a table local to the
master, to be
used as soon as the accumulator becomes available.
The \of{master} function distinguishes between \of{map} and \of{fold}
results and, accordingly, updates the table and the accumulator. For a
\of{map} result, either we combine it with the accumulator in a new
task or we simply store it in the local table. For a \of{fold} result,
either we simply store it into the accumulator or we immediately
combine it with a pending \of{map} result from the table.
Whenever \of{compute} returns, the accumulator must be available and
we simply return it.

\paragraph{\of{map_fold_ac}.}
The function \of{map_fold_ac} is easier than the previous one, since
we do not need to store more than one intermediate result. Indeed, as
soon as two intermediate results are available, we immediately combine
them using a \of{fold} operation. Thus, whenever the \of{master}
receives some result, it either stores it or combines it with the
accumulator, depending on the availability/non availability of the
accumulator.  Whenever \of{compute} returns, the accumulator must be
available and we simply return it.

\paragraph{\of{map_fold_a}.}
Function \of{map_fold_a} is definitely the trickiest. Indeed, the
operation being only associative, we can only combine \emph{adjacent}
intermediate results. In order to perform this, we store intermediate
results in a local table according to indices of the input list. More
precisely, if we have already computed
\begin{displaymath}
  x_i \oplus x_{i+1} \oplus \dots \oplus x_j
\end{displaymath}
where $\oplus$ denotes the \of{fold} operation, then we associate this
value with indices $i,j$. As soon as a result of indices $k,i-1$ or
$j+1,k$ is available, it can be combined with the result of $i,j$.
Whenever \of{compute} returns, the table should only contain a result
corresponding to the full input list and we simply return it.

\medskip
These five functions are the most natural derivations of the
\of{compute} functions. There are obviously other possible derivations
(e.g. a variant of \of{map_fold} where only \of{fold} is meaningful);
in most cases, they could be derived in a straightforward way from our
generic API.

% TODO cannot be instantiated in the monomorphic case

\section{Illustrative Case Studies}

\subsection{Matrix Multiplication}

As a first example, let us consider matrix multiplication.
We assume two matrices \of{a} and \of{b}, respectively of size
$\of{n}\times\of{p}$ and $\of{p}\times\of{m}$, given as input, as well as
a matrix \of{c} of size $\of{n}\times\of{m}$ to store the result.
Assuming \of{a} being a row-matrix and \of{b} a column-matrix,
the standard matrix multiplication would be as follows:
\begin{ocaml}
  for i = 0 to n-1 do
    for j = 0 to p-1 do
      for k = 0 to m-1 do
	c.(i).(j) <- c.(i).(j) + a.(i).(k) * b.(j).(k)
      done
    done
  done  
\end{ocaml}
and runs in $O(\of{n}\times\of{p}\times\of{m})$, assuming constant
time addition and multiplication (denoted $+$ and $\times$).
An easy way to distribute this computation is obviously to turn each
computation of the inner loop over \of{k} into a task.
To do so, we first prepare the list of tasks:
\begin{ocaml}
  let tasks = 
    let l = ref [] in
    for i = 0 to n-1 do for j = 0 to p-1 do
      tasks := ((a.(i), b.(j)), (i,j)) :: !tasks
    done done;
    !l
\end{ocaml}
Each task consists of row \of{a.(i)} and column \of{b.(j)} as first
tuple, together with position \of{(i,j)} as a second tuple.
The worker is receiving the first tuple as argument and computes
the dot product:
\begin{ocaml}
  let worker (ai, bj) =
    let c = ref 0 in
    for k = 0 to m-1 do c := !c + ai.(k) * bj.(k) done;
    !c
\end{ocaml}
The master is a one line function which receives the result \of{r}
from the worker and simply stores it into \of{c}, according to the
position contained in the input task. It produces no new task.
\begin{ocaml}
  let master (_, (i,j)) r = c.(i).(j) <- r; []
\end{ocaml}
Finally, the overall computation is started by invoking \of{compute}
as follows:
\begin{ocaml}
  let () = compute ~worker ~master !tasks
\end{ocaml}
The total number of tasks is $\of{n}\times\of{m}$.

Using the sequential implementation provided by \of{Functory} is as
simple as including the following line in the code:
\begin{ocaml}
  open Sequential
\end{ocaml}
Then the computation is roughly similar to the standard multiplication
described above. 

Now, let us assume we want to make use of a 4-core machine to perform
the same computation. 
This is achieved by just replacing the line above by
\begin{ocaml}
  open Cores
  let () = set_number_of_cores 4
\end{ocaml}
while the rest of the code remains unchanged.

Later, we may want to use a network of machines instead, for example
two machines called \texttt{orcus} and \texttt{belzebuth} with 4 and 8
cores respectively. Similar to the above, we turn the above lines into
\begin{ocaml}
  open Network
  let () = declare_workers ~n:4 "orcus"
  let () = declare_workers ~n:8 "belzebuth"
  open Same
\end{ocaml}
Here, \of{Same} is a module which is to be used when master and worker
are running the same executable. It provides a \of{compute} function
which has the same signature as in modules \of{Sequential} and
\of{Cores}, so that the rest of the code remains unchanged. Master and
worker processes are distinguished at run-time using an environment
variable \texttt{WORKER} which is set/unset.

If we need to write two different programs for master and worker, for
reason of binary incompatibility or any other reason, the library API
is providing functions to do that. If master and worker are still
compiled with the same version of \Ocaml, we use the \of{Poly} module
which provides a polymorphic API. Let us start with the worker program.
It now looks like:
\begin{ocaml}
  open Poly
  let worker (ai, bj) = ...
  let () = Worker.compute worker ()
\end{ocaml}
The \of{Worker.compute} function enters a loop which waits for
tasks sent by the master and returns results computed using \of{worker}.
The master program is almost the same as before. First, we replace
module \of{Same} with module \of{Poly}:
\begin{ocaml}
  open Network
  let () = declare_workers ~n:4 "orcus"
  let () = declare_workers ~n:8 "belzebuth"
  open Poly
\end{ocaml}
Tasks and \of{master} function are unchanged:
\begin{ocaml}
  let tasks = ...
  let master (_, (i,j)) r = ...
\end{ocaml}
Finally, we start the computation with \of{Master.compute}, which does
not have a \of{worker} parameter anymore:
\begin{ocaml}
  let () = Master.compute ~master tasks
\end{ocaml}

When master and worker programs are compiled with different
versions of the \Ocaml\ compiler, our library still provides
a monomorphic API over strings. As a consequence, we need to convert
tasks and results to and from strings in both master and worker.
The modified worker program then looks as follows:
\begin{ocaml}
  open Mono
  let worker (ai, bj) = ...
  let worker_string t = string_of_int (worker (task_of_string i))
  let () = Worker.compute worker_string ()
\end{ocaml}
The master program is modified in a similar way.
We simply replace \of{Poly} with \of{Mono} and encode/decode integers
as strings, as follows:
\begin{ocaml}
  let tasks = ... string_of_task ...
  let master (_, (i,j)) r = c.(i).(j) <- int_of_string r; []
\end{ocaml}
where \of{string_of_task} and \of{task_of_string} are user-defined
functions to convert tasks to and from strings.

\subsection{N-queens}\label{sec:n-queens}

The first example is the classical $N$-queens problem, where we
compute the total number of ways to place $N$ queens on a $N\times N$
chessboard in such a way no two queens attack each other.
We use a standard backtracking algorithm for this problem, which
places the queens one by one starting from the first row.
Distributing the computation is thus quite easy: we consider all
possible ways to place queens on the first $D$ rows and then perform
the subsequent search in parallel. Choosing $D=1$ will result in
exactly $N$ tasks. More subtly, $D=2$ will result in $N^2-3N+2$ tasks.
% TODO: make a picture to illustrate positioning of 2 queens on the
% first 2 rows

Each tasks only consists of three integers and results in one integer,
which is the total number of solutions for this task.
We make use of function \of{map_local_fold} from the derived API,
where \of{map} is performing the search and \of{fold} simply adds the
intermediate results.

\subsection{Mandelbrot Set}

\subsection{Polynomial Multiplication}

\section{Implementation Details}\label{sec:implem}

We now describe the implementations of the various modules introduced
in Section~\ref{sec:API}. The implementation of the \of{Sequential}
module is straightforward and does not require any explanation.

\subsection{Multiple Cores}

The \of{Cores} module implements the distributed computing library for
multiple cores on the same machine. 
% As illustrated in Section~\ref{sec:example}, 
It provides a function
\of{set_number_of_cores: int -> unit} to indicate the number of cores
to be used. The number passed to this function may be different from
the actual number of cores in the machine; it is rather the number of
tasks to be performed simultaneously.
% TODO comment more on the number of cores?
% Thus a program using the \of{Cores} module typically begins with 
% \begin{ocaml}
% open Cores
% let () = set_number_of_cores 8
% \end{ocaml}

The \of{Cores} module is implemented with \unix\ processes, using the
\of{fork} and \of{wait} system calls provided by the \of{Unix} library
of \Ocaml. The idea is pretty simple.  The \of{compute} function
maintains a global table of pending tasks and keeps track of the
number of idle cores.  Whenever there is a pending task and an idle
core, a new sub-process is created using \of{Unix.fork}; once
completed, the sub-process marshals the result into a 
local file. The
main process maintains a table mapping each sub-process ID to the
input task and the local file name. It waits for any completed
sub-process using \of{Unix.wait} and recovers the result from the
local file. Then function \of{master} is applied, which may generate
new tasks.  The main loop can be depicted in the following way:
\begin{flushleft}
  \quad  \textbf{while} pending tasks $\lor$ pending sub-processes \\
  \quad  \quad \textbf{while} pending tasks $\land$ idle cores \\
  \quad  \quad \quad create new sub-process for some task \\
  \quad  \quad \textbf{wait} for any completed sub-process \\
  \quad  \quad \quad push new tasks generated by \of{master} \\
\end{flushleft}
We also have an alternative implementation using
\unix\ pipes instead of local files.

The scheduling of tasks to the different cores is left to the
operating system, through \of{Unix.fork}. Thus it may be the case that
two tasks are executed on the same core, even if the declared number
of cores is less or equal than the actual number of cores on the machine.
% TODO: comment on user-provided scheduling?

\subsection{Network of Machines}

The \of{Network} module implements the distributed computing library
for a network of machines. 
% As illustrated in Section~\ref{sec:example}, 
It provides a function
\of{declare_workers: n:int -> string -> unit} to fill a table of
worker machines. 
% The argument \of{n} indicates the number of cores to
% be used in each worker machine, analogous to the \of{Cores}
% module. Thus a program using the \of{Network} module typically begins
% with
% \begin{ocaml}
% open Network
% let () = declare_workers ~n:12 "hostname1"
% let () = declare_workers ~n:8  "192.168.24.2"
% ...
% \end{ocaml}

The \of{Network} module is based on a traditional TCP-based client/server
architecture, where each worker is a server and the master is the
client of each worker. The main execution loop is similar to the one
in the \of{Cores} module, where distant processes on remote machines
correspond to sub-processes and idle cores are the idle cores 
of remote workers. In addition to this, it also involves issues of message
transfer and fault tolerance, which are subsequently described.

\subsubsection{Protocol}\label{sec:protocol}

Messages sent from master to workers could be any of the following kinds:
\begin{description}
\item[\of{Assign(id:int, f:string, x:string)}] This message assigns a
  new task to the worker, the task being identified by the unique
  integer \of{id}. The task to be performed is given by strings \of{f}
  and \of{x}, which are interpreted depending on the context.

\item[\of{Kill(id:int)}] This message tells the worker to kill the task
  identified by \of{id}.

\item[\of{Stop}] This message informs the worker about completion of
  the computation, so that it may choose to exit.

\item[\of{Ping}] This message is used to check if the worker is still
  alive, expecting a \of{Pong} message from the worker in return.
\end{description}
Messages sent by workers could be any of the following kinds:
\begin{description}
\item[\of{Pong}] This message is an acknowledgment for a \of{Ping}
  message from the master.

\item[\of{Completed(id:int, s:string)}] This message indicates the
  completion of a task identified by \of{id}, with result \of{s}.

\item[\of{Aborted(id:int)}] This message informs the master that the
  task identified by \of{id} is aborted, either as a response to a
  \of{Kill} message or because of a worker malfunction.
\end{description}
Our implementation of the protocol works across different
architectures, so that master and workers could be run on completely
different platforms w.r.t. endianness, version of \Ocaml\ and
operating system.

\subsubsection{Fault Tolerance}\label{sec:fault}

The main issue in any distributed computing environment is the ability
to handle faults, which is also a distinguishing feature of our
library.  Faults are mainly of two kinds: either a worker is stopped,
and possibly later restarted; or a worker is temporarily or
permanently unreachable on the network. To provide fault tolerance,
our master implementation is keeping track of the status of each
worker.  This status is controlled by two timeout parameters $T_1$ and
$T_2$ and \of{Ping} and \of{Pong} messages sent by master and workers,
respectively. There are four possible statuses for a worker:
\begin{description}
\item[\of{not connected}:] there is no ongoing TCP connection between
  the master and the worker;
\item[\of{alive}:] the worker has sent some message
  within $T_1$ seconds;
\item[\of{pinged}:] the worker has not sent any message within $T_1$
  seconds and the master has sent the worker a
  \of{Ping} message within $T_2$ seconds;
\item[\of{unreachable}:] the worker has not yet responded to the \of{Ping}
  message (for more than $T_2$ seconds).
\end{description}
% TODO picture
Whenever we receive a message from a worker, its status changes to
\of{alive} and its timeout value is reset.
\begin{center}
  \includegraphics{state.mps}
\end{center}

Fault tolerance is achieved by exploiting the status of workers as
follows. First, tasks are only assigned to workers with either
\of{alive} or \of{pinged} status. Second, whenever a worker executing
a task $t$ moves to status \of{not connected} or \of{unreachable}, the
task $t$ is rescheduled, which means it is put back in the set of
pending tasks. Whenever a task is completed, any rescheduled copy of
this task is either removed from the set of pending tasks or killed if
it was already assigned to another worker.

\subsubsection{Three Network Implementations} % TODO: change title

As mentioned in Section~\ref{sec:API}, the \of{Network} module
actually provides three different implementations, according to three
different execution scenarios (namely, \of{Same}, \of{Poly} and
\of{Mono}). There is
actually a single, generic implementation, which is then instantiated
in three different ways, corresponding to the three scenarios.  The
generic implementation assumes that only strings are being transmitted
over the network, as evident from the \of{Assign} and \of{Completed}
messages in Section~\ref{sec:protocol}. Depending on the
implementation, these strings could be marshaled closures, marshaled
values or simply user strings.

% The signature for the generic
% implementation is therefore as follows:
% \begin{ocaml}
%   generic_master:
%     assign_job:('a -> string * string) ->
%     master:('a * 'c -> string -> ('a * 'c) list)  ->
%     ('a * 'c) list -> unit
% \end{ocaml}
% The first argument \of{assign_job} is a function indicating the two
% strings that are passed in the \of{Assign} message of the protocol.
% We can now distinguish the three cases, based on \of{assign_job}:
% \begin{description}
% \item[\of{Same}:] in this case, \of{assign_job} simply returns
%   the marshaled closure together with the marshaled value.
% \item[\of{Poly}:] in this case, \of{assign_job} is simply
%   returning the marshaled value (the first string being meaningless in
%   this context).
% \item[\of{Mono}:] in this case, \of{assign_job} is immediately
%   returning its argument (the first string being meaningless in
%   this context).
% \end{description}

\section{Experiments}\label{sec:experiments}

Our library, \functory, is implemented in \Ocaml\ and is available
from \url{http://www.lri.fr/~filliatr/functory/}.
In this section, we show the potential of this library using two 
computationally extensive examples.

\subsubsection{Matrix Multiplication}

$\of{n} = \of{p} = \of{m} = 100$

coefficients have several thousands of digits; we use GMP

%            parall  line

% # tasks    10,000   100

% sequential  20.3   20.2
% cores 2     22.7   11.3
% cores 4     12.3    6.1
% cores 6      8.6    4.3
% cores 8      8.0    3.5

% network is localhost (moloch) + belzebuth / same number of workers on each

% network 3+3 142
% network 6+6  31.7

\begin{center}
  \begin{tabular}{|l|l|l|}
    \hline
    sequential & cores & network \\\hline
    20.3 & 
  \end{tabular}
\end{center}

\subsection{N-queens}

The following table shows execution times for various values of $N$
and our three different implementations: \of{Sequential}, \of{Cores}, and
\of{Network}. The purpose of this experiment is to measure the speedup
w.r.t. the sequential implementation. Therefore, all
computations are performed on the same machine, a 8 cores Intel Xeon
3.2 GHz running Debian Linux. The sequential implementation uses a
single core. The multi-core implementation uses the 8 cores of
the machine. The network implementation uses 8 workers running locally and
a master running on a remote machine (which incurs communication cost).

The first column shows the value of $N$.  The number of tasks is shown in
second column.  Then the last three columns show execution times in
seconds for the three implementations. The figures within brackets
show the speedup w.r.t. sequential implementation.
\begin{center}
  \begin{tabular}{|r|r|r|r|r|}
    \hline
    N & \#tasks  & sequential& cores                 & network 
    \\\hline\hline
    16 &   16    &  15.2     &   2.04 (7.45$\times$) &  2.35  (6.47$\times$) 
    \\\hline
       &  210    &  15.2     &   2.01 (7.56$\times$) & 21.80  (0.69$\times$)
    \\\hline
    17 &   17    & 107.0     &  17.20 (6.22$\times$) & 16.20  (6.60$\times$)
    \\\hline
       &  240    & 107.0     &  14.00 (7.64$\times$) & 24.90  (4.30$\times$)
    \\\hline
    18 &   18    & 787.0     & 123.00 (6.40$\times$) & 125.00 (6.30$\times$)  
    \\\hline
       &  272    & 787.0     & 103.00 (7.64$\times$) & 124.00 (6.34$\times$)  
    \\\hline
    19 &   19    &6120.0     & 937.00 (6.53$\times$) & 940.00 (6.51$\times$)  
    \\\hline
       &  306    &6130.0     & 796.00 (7.70$\times$) & 819.00 (7.48$\times$)
    \\\hline
  \end{tabular}
\end{center}
From the table above, it is clear that the \of{Cores} and \of{Network}
implementations provide a significant speedup. As evident from the
last row, the speedup is almost 8, which is also the number of
cores we use.  It is also evident from the last column that the
\of{Network} implementation performs significantly better when the
computation time dominates in the total execution time.  The two extreme
cases correspond to the second and the last row: in the second row, the
communication time dominates and is in fact more than 91\%\ of the
total execution time; on the other hand, for the last row
communication time amounts to just 4.6\%\ of the total execution time.
As expected, the network implementation is only beneficial when the
computation time for each individual task is significant, which is the
case in realistic examples.

\subsection{SMT Solvers}

Here we demonstrate the potential of our library for our application
needs as mentioned in the introduction. Our computing
infrastructure consists of 3 machines with 4, 8 and 8 cores
respectively. The benchmark we use for
experimentation consists of around 80 challenging verification conditions
(VC) obtained from the Why platform~\cite{filliatre07cav}.  Each
VC is stored in a file, which is accessible over
NFS. The purpose of the experiment is to check the validity of each VC
using several automated provers (namely Alt-Ergo, Simplify, Z3 and CVC3),
which are all installed on each of the machines in use.

The master program proceeds by reading the file names, turning them
into tasks by multiplying them by the number of provers (more than
300 tasks in total).
Each worker in turn invokes the given prover on the given file.
We set a timeout limit of 1 minute.
Each task completes with one of the four possible outcomes: \emph{valid},
\emph{unknown} (depending on
whether the VC is valid or undecided by the prover), 
\emph{timeout} and \emph{failure}.
The figure below shows the total time in minutes spent by each prover
for each possible outcomes.
\begin{center}
  \begin{tabular}{|r||r|r|r|r|}
    \hline
    prover   & valid & unknown & timeout & failure
    \\\hline\hline
    Alt-ergo & 406.0 & 3.0   &  11400.0 & 0.0       
    \\\hline
    Simplify &  0.5   & 0.4   &  1200.0 & 222.0   
    \\\hline
    Z3       & 80.7   & 0.0   &  1800.0 & 1695.0   
    \\\hline
    CVC3     & 303.0  & 82.7  &  4200.0 & 659.0   
    \\\hline
  \end{tabular}
\end{center}
These figures sum up to more than 6 hours if provers were executed
sequentially. However, using our library and our 3-machine
infrastructure, it completes in 22 minutes and 37 seconds, giving us a
speedup of more than 16$\times$. We are still far away from the ideal
ratio of 20$\times$ (we are using 20 cores), since some provers are
allocating a lot of memory and time spent in system calls is not
accounted for in the total observed time. However, a ratio of
16$\times$ is already a significant improvement for our day-to-day
experiments. 

\section{Conclusion and Future Work}\label{sec:future}

We presented a distributed programming environment for functional
programming. The main features are the genericity of the interface,
which makes use of polymorphic higher-order functions, and the ability
to easily switch between sequential, multi-core, and network
implementations. A distinguishing feature of our library is a
robust fault-tolerance mechanism which relieves
the user of cumbersome implementation details.

Closest to the approach in this paper is Yohann Padioleau's MapReduce
implementation in \Ocaml~\cite{poor-man-mapreduce}.  It is built on
top of OCamlMPI~\cite{ocamlMPI}, while our approach uses a homemade
protocol for message passing.  Currently, we have less flexibility
w.r.t. deployment of the user program than OCamlMPI; on the other
hand, we provide a 
more generic API together with fault tolerance.  There are other
distributed computing libraries on top of which one could implement
the library discussed in this paper. \JoCaml~\cite{jocaml} is one of
them. However, \JoCaml\ does not provide fault tolerance, which is
indispensable in a distributed setting. The user has to include code
for fault tolerance, as already demonstrated in some \JoCaml\
experiments~\cite{mandel2008}. 

There are other implementations of distributed computing in the
context of functional programming. One is the Disco
project~\cite{disco}, which implements exactly Google's MapReduce in
Erlang~\cite{erlang}. Our library, on the contrary, is not an \Ocaml\
implementation of Google's MapReduce.
There are other ways to exploit multi-core architectures. One of these
is data parallelism, which is also relevant in the functional
programming setting~\cite{parallel-haskell}. Our work
does not target data parallelism at all.

There are still some interesting features that could be added to our
library. 
\begin{itemize}
\item 
  One is the ability to efficiently assign tasks to workers depending
  on resource parameters, such as data locality, CPU power, memory,
  etc. This could be achieved by providing the user with the means to control
  task scheduling.  
\item 
  Another interesting feature could be the ability
  to add or remove workers dynamically. Currently, our library assumes
  that the list of workers to be used is given \emph{a priori}.
\item
  Our library provides limited support for displaying real-time
  information about the computations. Processing and storing
  information about tasks locally in the master is straightforward; 
  monitoring it in real-time is less obvious.
\item 
  One very nice feature of Google's MapReduce is the possibility to
  use redundantly several idle workers on the same tasks
  for speedup when reaching the end of computation.
  Since we already have the fault tolerance implemented, this
  optimization should be straightforward to add to our library.
\end{itemize}
We intend to enrich our library with all above features.

%\appendix


\paragraph{Acknowledgements.}
The authors are grateful to the ProVal team for support and comments
on early versions of the library and of this paper.

%\vfill\pagebreak
% We recommend abbrvnat bibliography style.
\bibliographystyle{abbrvnat}
\bibliography{./biblio}

\end{document}

% LocalWords:  parallelize functor parameterized indices endianness monomorphic
% LocalWords:  genericity executables OCamlMPI MapReduce
