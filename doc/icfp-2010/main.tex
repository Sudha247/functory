%-----------------------------------------------------------------------------
%
%               Template for sigplanconf LaTeX Class
%
% Name:         sigplanconf-template.tex
%
% Purpose:      A template for sigplanconf.cls, which is a LaTeX 2e class
%               file for SIGPLAN conference proceedings.
%
% Author:       Paul C. Anagnostopoulos
%               Windfall Software
%               978 371-2316
%               paul@windfall.com
%
% Created:      15 February 2005
%
%-----------------------------------------------------------------------------


\documentclass[preprint]{sigplanconf}

% The following \documentclass options may be useful:
%
% 10pt          To set in 10-point type instead of 9-point.
% 11pt          To set in 11-point type instead of 9-point.
% authoryear    To obtain author/year citation style instead of numeric.

\usepackage{amsmath,graphicx, color}

\newcommand{\Ocaml}{OCaml}
\newcommand{\unix}{\textsc{Unix}}

\begin{document}

\conferenceinfo{ICFP '10}{September 27-29, 2010, Baltimore, Maryland.} 
\copyrightyear{2010} 
\copyrightdata{[to be supplied]} 

\titlebanner{submitted to ICFP 2010}        % These are ignored unless
\preprintfooter{Experience Report: Distributed Computing for the Common Man}   % 'preprint' option specified.

\title{Experience Report: Distributed Computing for the Common Man}
%\subtitle{Subtitle Text, if any}

\authorinfo{Jean-Christophe Filli�tre}
           {CNRS, LRI, Univ Paris-Sud, Orsay F-91405\\
            INRIA Saclay - �le-de-France, ProVal, Orsay, F-91893}
           {filliatr@lri.fr}
\authorinfo{Kalyan Krishnamani}
           {INRIA Saclay - �le-de-France, ProVal, Orsay, F-91893}
           {kalyan@lri.fr}

\maketitle

\begin{abstract}
This is the text of the abstract.
\end{abstract}

\category{CR-number}{subcategory}{third-level}

\terms
term1, term2

\keywords
keyword1, keyword2

\section{Introduction}

This paper introduces a generic distributed computing library. This
work was initially motivated by the computing needs that exist in our
own research team. Our applications include large-scale deductive
program verification, which amounts to checking the validity of a
large number of logical formulas using a variety of automated theorem
provers~\cite{filliatre07cav}. Our computing infrastructure consists
of a few powerful multi-core machines (typically 8 to 16 cores) and
several desktop PCs (typically dual-core). However, for our
application needs, there is no library that helps in exploiting such a
computing infrastructure in our favorite functional programming
language.  Hence we designed and implemented such a
library\footnote{This library is implemented in \Ocaml, but the
  implementation should be straightforward in any functional
  programming language.}, which is the subject of this paper.



The distributing computing library presented in this paper is not a
library that helps in parallelizing computations. Rather, it provides
facilities for reliable, distributed execution of parallelizable
computations. In particular, it provides a set of APIs that allows the
execution of large-scale parallelizable computations, very relevant to
our application needs (and presumably relevant to a variety of
real-world applications), over multiple cores in the same machine or
over a network of machines. The most important features of our library
are the following:
\begin{itemize}
\item \emph{genericity}: 
  it allows various patterns of polymorphic computations;
\item \emph{simplicity}: switching between multiple cores on the same
  machine and a network of machines is as simple as changing a couple
  of lines of code;
\item \emph{task distribution and fault-tolerance}: 
  it relieves the user from implementing task distribution routines
  and reliable fault-tolerance mechanisms.
\end{itemize}
It is worth noting that the library is not targeted at applications
running on server farms, crunching enormous amounts of data. Neither
is it limited to research labs possessing a few number of powerful
computing infrastructure. It serves a variety of users and a wide
spectrum of needs, from desktop PCs to networks of machines, and hence
the title.


The application domain of such a distributed computing library is
manyfold.  The most striking application, as also relevant to our
research endeavours, is in validating thousands of verification
conditions using automated theorem provers. Having noted this, it
should be apparent that the library finds use in efficient execution
of any large-scale expensive computation that is parallelizable.


% what it is and what it is not
% \begin{itemize}
% \item it is not a library which does parallelization
% \item it is a library for reliable distributed execution of
%   parallelizable computation 
% \end{itemize}

% inspired by Google's MapReduce~\cite{mapreduce} (itself inspired by functional
% programming, ironically) ;  however, there are differences:
% \begin{itemize}
% \item polymorphic, higher-order API
% \item more generic (cores/network)
% \item does not focus on association lists
% \item we don't have data locality (future work)
% \end{itemize}

% what is the target audience/applications: 
% for instance, use if automatic provers on thousands of verification
% conditions, on computing infrastructures which can be
% \begin{itemize}
% \item a single machine with multiple cores, possibly remote
% \item several machines, small or large in number, over a network
% \end{itemize}

\paragraph{Distributed Computing.}
A typical distributed computing library provides the following (we
borrow some terminology from Google's Mapreduce~\cite{mapreduce}):
\begin{itemize}
\item A notion of a \emph{task} which denotes the atomic computation to be
  performed in a distributed manner;
\item A set of machines or programs called \emph{workers} that perform
  the tasks, producing results;
\item A single program called a \emph{master} that is in charge
  of distributing the tasks among the workers and managing results
  produced by the workers.
\end{itemize}
In addition to the above, distributed computing environments also
implement mechanisms for fault-tolerance, efficient storage and
distribution of tasks. This is required to handle network failures
that may occur, as well as to optimize the usage of machines in the
network. Another concern of importance is the transmission of messages
over the network. This requires efficient serialization (or
marshalling) of data that can be reliably reconstructed when
transmitted over different computing environments.  It is desirable to
maintain architecture independence while transmitting marshalled data,
as machines in a distributed computing environment often run on
different hardware architectures and make use of varying software
platforms. For example, endianness may be different across machines on
the network.

\paragraph{A Functional Programming Approach.}
Our work was initially inspired by Google's
Mapreduce\footnote{Ironically, Google's approach itself was inspired
  by functional programming primitives.}. However, our functional
programming environment allows us to be more generic. 
The main idea behind our approach is that
workers may implement any polymorphic function
\begin{ocaml}
  worker: 'a -> 'b
\end{ocaml}
where \of{'a} denotes the type of tasks and \of{'b} the type of results.
Then the master is given a list of initial tasks, together with a
function to handle the results:
\begin{ocaml}
  master: ('a -> 'b -> 'a list) -> 'a list -> unit
\end{ocaml}
The function passed to the master is applied whenever a result is
available and may in turn generate new tasks (hence the return type
\of{'a list}).  The master is executed as long as there are incomplete
tasks.

Our library makes use of \Ocaml's marshaling capabilities, as much as
possible. Whenever master and worker executables are exactly the same,
we can marshal polymorphic values and closures. However, it is not
always possible to have master and workers running the same
executable. In this case, we cannot marshal closures anymore but we
can still marshal polymorphic values as long as the same version of
\Ocaml\ is used to compile master and workers. When different versions
of \Ocaml\ are used, we can no longer marshal values but we can still
transmit strings between master and workers. Our library adapts to all
these situations, by providing several APIs.


The paper is organized as follows. In Section~\ref{sec:API}, we
describe the main idea behind the generic API, and introduce
\of{master} and \of{worker} programs. In Section~\ref{sec:derived}, we
derive more interesting functions from the generic
API. Section~\ref{sec:implem} delves into the implementation details
of our library, while Section~\ref{sec:experiments} illustrates the
potential of the presented library through experimental evaluation. We
compare our approach with relevant related work in
Section~\ref{sec:future} and outline our future work.

\section{API}\label{sec:API}

The main function in our API follows the idea sketched in the
introduction. It has the following signature:
\begin{ocaml}
  val compute : 
    worker:('a -> 'b) -> 
    master:('a * 'c -> 'b -> ('a * 'c) list) -> 
    ('a * 'c) list -> unit
\end{ocaml}
Tasks are pairs, of type \of{'a * 'c}, where the first component is
passed to the worker and the second component is local to the master.
The function \ocaml{worker} should be pure and is executed in parallel
with other \ocaml{worker}s. The function \ocaml{master}, on the
contrary, can be impure and is never executed in parallel with other
\ocaml{master}s. The function \ocaml{master} always processes
information locally, typically storing it to internal data structures.
The next section will describe the usage of this generic interface to
perform traditional map and fold operations.

Actually, our library provides not just a single \ocaml{compute}
function as above, but instead five different versions
depending on the execution context. 
The five possible contexts are the following: 
\begin{enumerate}
\item \textbf{purely sequential execution:}
  this is intended for validating the results;

\item \textbf{several cores on the same machine:} 
  this implementation makes use of \unix\ processes and thus can
  provide a polymorphic \ocaml{compute} function;

\item \textbf{same executable run on master and worker machines:}
  this implementation makes use of the ability to marshal \Ocaml\
  closures and thus can still provide a polymorphic \ocaml{compute} function;

\item \textbf{master and workers are different programs, compiled with
    the same version of \Ocaml:} 
  we can no longer marshal closures but we can still
  marshal polymorphic values. As a consequence, 
  the \ocaml{compute} function is split into two
  polymorphic functions, to implement the master and workers 
  respectively:%
\begin{ocaml}
val master : 
  ('a * 'c -> 'b -> ('a * 'c) list) -> 
  ('a * 'c) list -> unit
val worker : ('a -> 'b) -> unit -> unit
\end{ocaml}
\item \textbf{master and workers are different programs, not even
    compiled with the same version of \Ocaml:} we can no
  longer use marshaling, so the
  \ocaml{compute} function is split into two monomorphic functions
  over strings:%
\begin{ocaml}
val master : 
  (string * 'c -> string -> (string * 'c) list) -> 
  (string * 'c) list -> unit
val worker : (string -> string) -> unit -> unit
\end{ocaml}
  Note that the second component of each task is still polymorphic (of
  type \ocaml{'c} here), since it is local to the master.
\end{enumerate}

Our library is organized into three modules: \of{Sequential} for the
pure sequential implementation, \of{Cores} for multiple cores on the
same machine and \of{Network} for a network of machines, respectively.
The \of{Network} module itself is organized into three sub-modules,
corresponding to contexts 3, 4 and 5 above. The different
API functions are summarized in the following table (their signatures
being given above):
\begin{center}
  \begin{tabular}{|c|c|c|c|c|}
    \hline
    \ocaml{Sequential} & \ocaml{Cores} &
    \multicolumn{3}{|c|}{\ocaml{Network}} 
    \\\cline{3-5}
    &       & \ocaml{Same} & \ocaml{Poly} & \ocaml{Mono} \\\hline
    \ocaml{compute}  & \ocaml{compute}   & \ocaml{compute}  &
    \ocaml{master}  & \ocaml{master}  \\
    & & & \ocaml{worker}  & \ocaml{worker} \\\hline
  \end{tabular}
\end{center}
The following section introduces other interesting APIs that can be
derived from this generic API.

\section{Derived API}\label{sec:derived}

In most cases, the easiest way to parallelize an execution it to make
use of map and fold operations over lists, where processing of the
list elements are done in parallel.  To facilitate such a processing,
we now derive the most commonly used list operations from our generic
API.

The most obvious operation is the traditional map operation over
lists, that is:
\begin{ocaml}
  val map : ('a -> 'b) -> 'a list -> 'b list
\end{ocaml}
The next natural operation is a combination of map and fold
operations, that is a function like
\begin{ocaml}
 val map_fold :
   map:('a -> 'b) -> fold:('c -> 'b -> 'c) -> 
   'c -> 'a list -> 'c
\end{ocaml}
which, given two functions, an accumulator \of{a} and a list \of{l}, computes
\begin{equation}\label{eq:map-fold}
  \of{fold} ... (\of{fold} (\of{fold} ~ a ~ (\of{map} ~ x_1)) (\of{map} ~ x_2))
  ... (\of{map} ~ x_n)
\end{equation}
for some permutation $[x_1,x_2,...,x_n]$ of the list \of{l}.
We assume that the \of{map} operations are always performed in parallel.
Regarding \of{fold} operations, we distinguish two cases:
\begin{itemize}
\item either \of{fold} operations are computationally less expensive
  than \of{map} and we perform them locally;
\item or \of{fold} operations are computationally expensive and we
  perform them in parallel.
\end{itemize}
We thus provide two functions \of{map_local_fold} and \of{map_remote_fold}.

In the case of \of{map_remote_fold}, only one \of{fold} operation can
be performed at a time (possibly in parallel with \of{map}
operations), as obvious from~(\ref{eq:map-fold}). However, there are
cases where several \of{fold} operations can be performed in parallel,
as early as intermediate results of \of{fold} operations are available.
This is the case when \of{fold} is an associative operation (which
implies that type \of{'b} and \of{'c} are the same  and that the
accumulator should be a neutral element for \of{fold}). Whenever
\of{fold} is also commutative, we can perform even more \of{fold}
operations in parallel. Thus our API provides two functions
\of{map_fold_a} and \of{map_fold_ac} for these two particular cases.
The five operations of the derived API are summarized
in Figure~\ref{fig:derived}.
\begin{figure}[t]
  \begin{ocaml}
    val map : 
      f:('a -> 'b) -> 'a list -> 'b list 
    val map_local_fold : 
      map:('a -> 'b) -> fold:('c -> 'b -> 'c) -> 
      'c -> 'a list -> 'c 
    val map_remote_fold : 
      map:('a -> 'b) -> fold:('c -> 'b -> 'c) -> 
      'c -> 'a list -> 'c 
    val map_fold_ac : 
      map:('a -> 'b) -> fold:('b -> 'b -> 'b) -> 
      'b -> 'a list -> 'b 
    val map_fold_a : 
      map:('a -> 'b) -> fold:('b -> 'b -> 'b) -> 
      'b -> 'a list -> 'b
  \end{ocaml}
  \caption{Derived API.}
\label{fig:derived}
\end{figure}

These five functions are actually derived from the generic API.  The
generality is achieved by implementing these derivations as a functor.
This functor, \of{Derive}, is parameterized by the \of{compute}
function, as follows:
\begin{ocaml}
module Derive
  (X : sig
     val compute : 
       worker:('a -> 'b) -> 
       master:('a * 'c -> 'b -> ('a * 'c) list) ->
       ('a * 'c) list -> unit
   end) = struct ... end
\end{ocaml}
We now explain how each function from Figure~\ref{fig:derived} is
implemented using \of{compute} inside the functor body.

\paragraph{\of{map}.} 
Function \of{map} is easily implemented by associating an integer
index with each element (of type \of{'a}) of the input list. Thus the
list of tasks passed to \of{compute} is a list of pairs of type \of{'a * int}.
As soon as an intermediate result is available, it is passed to
\of{master}, which stores it in a local table using the index as a key
and generates no new task. Whenever \of{compute} returns, we easily
recover the output list from the values stored in the table.

\paragraph{\of{map_local_fold}.} 
Function \of{map_local_fold} is implemented using a local reference
storing the current accumulator (of type \of{'c}).  As soon as an
intermediate result is available, the \of{master} combines it with the
accumulator using the function \of{fold}. Whenever \of{compute}
returns, we simply return the current accumulator.

\paragraph{\of{map_remote_fold}.}
In this case, combining the accumulator with an intermediate result is
itself a task to be performed in parallel with \of{map} operations.
Thus, when an intermediate \of{map} result is available, the
accumulator may be already in use in a \of{fold} operation. Hence we
need to store the intermediate \of{map} result in a local table, to be
used as soon as the accumulator becomes available.
The \of{master} function distinguishes between \of{map} and \of{fold}
results and, accordingly, updates the table and the accumulator. For a
\of{map} result, either we combine it with the accumulator in a new
task or we simply store it in the local table. For a \of{fold} result,
either we simply store it into the accumulator or we immediately
combine it with a pending \of{map} result from the table.
Whenever \of{compute} returns, the accumulator must be available and
we simply return it.

\paragraph{\of{map_fold_ac}.}
The function \of{map_fold_ac} is easier than the previous one, since
we do not need to store more than one intermediate result. Indeed, as
soon as two intermediate results are available, we immediately combine
them using a \of{fold} operation. Thus the \of{master} function either
stores it or combines it, depending on the availability/non
availability of the accumulator.  Whenever \of{compute} returns, the
accumulator must be available and we simply return it.

\paragraph{\of{map_fold_a}.}
Function \of{map_fold_a} is definitely the trickiest. Indeed, the
operation being only associative, we can only combine \emph{adjacent}
intermediate results. In order to perform this, we store intermediate
results in a local table according to indices of the input list. More
precisely, if we have already computed
\begin{displaymath}
  x_i \oplus x_{i+1} \oplus \dots \oplus x_j
\end{displaymath}
where $\oplus$ denotes the \of{fold} operation, then we associate this
value with indices $i,j$. As soon as a result of indices $k,i-1$ or
$j+1,k$ is available, it can be combined with the result of $i,j$.
Whenever \of{compute} returns, the table should only contain a result
corresponding to the full input list and we simply return it.


These five functions are the most natural derivations of the
\of{compute} functions. There are obviously other possible derivations
(e.g. a variant of \of{map_fold} where only \of{fold} is meaningful);
in most cases, they could be derived in a straightforward way from our
generic API.

% TODO cannot be instantiated in the monomorphic case

\section{Implementation Details}\label{sec:implem}

We now describe the implementations of the various modules introduced
in Section~\ref{sec:API}. The implementation of the \of{Sequential}
module is straightforward and does not require any explanation.

\subsection{Multiple Cores}

The \of{Cores} module implements the distributed computing library for
multiple cores on the same machine. It provides a function
\of{set_number_of_cores: int -> unit} to indicate the number of cores
to be used. The number passed to this function may be different from
the actual number of cores in the machine; it is rather the number of
tasks to be performed simultaneously.
% TODO comment more on the number of cores?
Thus a program using the \of{Cores} module typically begins with 
\begin{ocaml}
open Mapreduce.Cores
let () = set_number_of_cores 8
\end{ocaml}
% TODO rename Mapreduce?

The \of{Cores} module is implemented with \unix\ processes, using the
\of{fork} and \of{wait} system calls provided by the \of{Unix} library
of \Ocaml. The idea is pretty simple.  The \of{compute} function
maintains a global table of pending tasks and keeps track of the
number of idle cores.  Whenever there is a pending task and an idle
core, a new sub-process is created using \of{Unix.fork}; once
completed, the sub-process marshals the result into a local file. The
main process maintains a table mapping each sub-process ID to the
input task and the local file name. It waits for any completed
sub-process using \of{Unix.wait} and recovers the result from the
local file. Then function \of{master} is applied, which may generate
new tasks.  The main loop can be depicted in the following way:
\begin{flushleft}
  \quad  \textbf{while} pending tasks $\lor$ pending sub-processes \\
  \quad  \quad \textbf{while} pending tasks $\land$ idle cores \\
  \quad  \quad \quad create new sub-process for some task \\
  \quad  \quad \textbf{wait} for any completed sub-process \\
  \quad  \quad \quad push new tasks generated by \of{master} \\
\end{flushleft}

It would be more efficient to use \unix\ pipes instead of local files
but pipes are severely restricted w.r.t. data size. To be more
generic, we make no assumptions about the size of the data. Providing
an alternative implementation using pipes would be straightforward.

The scheduling of tasks to the different cores is left to the
operating system, through \of{Unix.fork}. Thus it may be the case that
two tasks are executed on the same core, even if the declared number
of cores is less or equal than the actual number of cores on the machine.
% TODO: comment on user-provided scheduling?

\subsection{Network of Machines}

The \of{Network} module implements the distributed computing library
for a network of machines.  It provides a function
\of{declare_workers: n:int -> string -> unit} to fill a table of
worker machines. The argument \of{n} indicates the number of cores to
be used in each worker machine, analogous to the \of{Cores}
module. Thus a program using the \of{Network} module typically begins
with
\begin{ocaml}
open Mapreduce.Network
let () = declare_workers ~n:12 "machine1"
let () = declare_workers ~n:8  "machine2"
...
\end{ocaml}

The \of{Network} module is based on a traditional client/server
architecture, where each worker is a server and the master is the
client of each worker. The main execution loop is similar to the one
in the \of{Cores} module, where distant processes on remote machines
correspond to sub-processes and idle workers correspond to idle
cores. In addition to this, it also involves issues of message
transfer and fault tolerance, which are subsequently described.

\subsubsection{Protocol}\label{sec:protocol}

Messages sent from master to workers could be any of the following kinds:
\begin{description}
\item[\of{Assign(id:int, f:string, x:string)}] This message assigns a
  new task to the worker, the task being identified by the unique
  integer \of{id}. The task to be performed is given by strings \of{f}
  and \of{x}, which are interpreted depending on the context.

\item[\of{Kill(id:int)}] This message indicates the worker that task
  identified by \of{id} is to be killed.

\item[\of{Stop}] This message informs the worker about completion of
  the computation, so that it may choose to exit.

\item[\of{Ping}] This message is used to check if the worker is still
  alive, expecting a \of{Pong} message from the worker in return.
\end{description}
Messages sent by the workers could be any of the following kinds:
\begin{description}
\item[\of{Pong}] This message is an acknowledgment for a \of{Ping}
  message from the master.

\item[\of{Completed(id:int, s:string)}] This message indicates the
  completion of a task identified by \of{id}, with result \of{s}.

\item[\of{Aborted(id:int)}] This message informs the master that the
  task identified by \of{id} is aborted, either as a response to a
  \of{Kill} message or because of a worker malfunction.
\end{description}
Our implementation of the protocol works across different
architectures, so that master and workers could be run on completely
different platforms w.r.t. endianness, version of \Ocaml\ and
operating system.

\subsubsection{Fault Tolerance}\label{sec:fault}

The main issue in any distributed computing environment is the ability
to handle faults, which is also a distinguishing feature of our
library.  Faults are mainly of two kinds: either a worker is stopped,
and possibly later restarted; or a worker is temporarily or
permanently unreachable on the network. To provide fault tolerance,
our master implementation is keeping track of the status of each
worker.  This status is controlled by two timeout parameters $T_1$ and
$T_2$ and \of{Ping} and \of{Pong} messages sent by master and workers,
respectively. There are four possible statuses for a worker:
\begin{description}
\item[\of{not connected}:] the worker is still to be
  connected to perform any useful task;
\item[\of{alive}:] the worker has sent some message recently,
  within $T_1$ seconds;
\item[\of{pinged}:] the worker has not sent any message recently (for
  more than $T_1$ seconds) and the master has sent the worker a
  \of{Ping} message within the last $T_2$ seconds;
\item[\of{unreachable}:] the worker has not yet responded to the \of{Ping}
  message (for more than $T_2$ seconds).
\end{description}
% TODO picture
Whenever we receive a message from a worker, its status changes to
\of{alive} and its timeout value is reset.
\begin{center}
  \includegraphics{state.mps}
\end{center}

Fault tolerance is achieved by exploiting the status of workers as
follows. First, tasks are only assigned to workers with either
\of{alive} or \of{pinged} status. Second, whenever a worker executing
a task $t$ moves to status \of{not connected} or \of{unreachable}, the
task $t$ is rescheduled, which means it is put back in the set of
pending tasks. Whenever a task is completed, any rescheduled copy of
this task is either removed from the set of pending tasks or killed if
it was already assigned to an idle worker.

\subsubsection{Three Network Implementations} % TODO: change title

As mentioned in Section~\ref{sec:API}, the \of{Network} module
actually provides three different implementations, according to three
different execution scenarios (namely, \of{Same}, \of{Poly} and
\of{Mono}). The exciting thing about our library is that there is
actually a single, generic implementation, which is then instantiated
in three different ways, corresponding to the three scenarios.  The
generic implementation assumes that only string are being transmitting
over the network, as evident from \of{Assign} and \of{Completed}
messages in Section~\ref{sec:protocol}.  The signature for the generic
implementation is therefore as follows:
\begin{ocaml}
  generic_master:
    assign_job:('a -> string * string) ->
    master:('a * 'c -> string -> ('a * 'c) list)  ->
    ('a * 'c) list -> unit
\end{ocaml}
The first argument \of{assign_job} is a function indicating the two
strings that are passed in the \of{Assign} message of the protocol.
We can now distinguish the three cases, based on \of{assign_job}:
\begin{description}
\item[\of{Same}:] in this case, \of{assign_job} is simply returning
  the marshaled closure together with the marshaled value.
\item[\of{Poly}:] in this case, \of{assign_job} is simply
  returning the marshaled value (the first string being meaningless in
  this context).
\item[\of{Mono}:] in this case, \of{assign_job} is immediately
  returning its argument (the first string being meaningless in
  this context).
\end{description}

\section{Experiments}\label{sec:experiments}

n-queens, SMT solvers, etc.

\section{Future Work}\label{sec:future}

A couple of caveats are in order. (1) We do not focus on issues like
data-locality or a distributed file system like Google's, rather we
assume a \unix\ Network File System (NFS). (2) Our focus is not
exactly on Google's Mapreduce paradigm~\cite{mapreduce} that works on
association lists, but we also provide an API similar to Google's that
works on association lists, just to satisfy popular interest.

%TODO
data locality: scheduling supplied by the user

speeding up the end of computation with duplicate tasks on idle workers
(straightforward since already done for fault tolerance)

% \appendix
% \section{Appendix Title}

% This is the text of the appendix, if you need one.

\acks

Acknowledgments, if needed.

% We recommend abbrvnat bibliography style.

\nocite{*}
\bibliographystyle{abbrvnat}
\bibliography{./biblio}

\end{document}

% LocalWords:  parallelize functor parameterized indices endianness monomorphic
% LocalWords:  genericity executables
