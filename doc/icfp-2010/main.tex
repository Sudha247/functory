%-----------------------------------------------------------------------------
%
%               Template for sigplanconf LaTeX Class
%
% Name:         sigplanconf-template.tex
%
% Purpose:      A template for sigplanconf.cls, which is a LaTeX 2e class
%               file for SIGPLAN conference proceedings.
%
% Author:       Paul C. Anagnostopoulos
%               Windfall Software
%               978 371-2316
%               paul@windfall.com
%
% Created:      15 February 2005
%
%-----------------------------------------------------------------------------


\documentclass[preprint]{sigplanconf}

% The following \documentclass options may be useful:
%
% 10pt          To set in 10-point type instead of 9-point.
% 11pt          To set in 11-point type instead of 9-point.
% authoryear    To obtain author/year citation style instead of numeric.

\usepackage{amsmath}

\begin{document}

\conferenceinfo{ICFP '10}{September 27-29, 2010, Baltimore, Maryland.} 
\copyrightyear{2010} 
\copyrightdata{[to be supplied]} 

\titlebanner{banner above paper title}        % These are ignored unless
\preprintfooter{short description of paper}   % 'preprint' option specified.

\title{Experience Report: }
\subtitle{Subtitle Text, if any}

\authorinfo{Name1}
           {Affiliation1}
           {Email1}
\authorinfo{Name2\and Name3}
           {Affiliation2/3}
           {Email2/3}

\maketitle

\begin{abstract}
This is the text of the abstract.
\end{abstract}

\category{CR-number}{subcategory}{third-level}

\terms
term1, term2

\keywords
keyword1, keyword2

\section{Introduction}

what it is and what it is not
\begin{itemize}
\item it is not a library which does parallelization
\item it is a library for reliable distributed execution of
  parallelizable computation 
\end{itemize}

inspired by Google's MapReduce~\cite{mapreduce} (itself inspired by functional
programming, ironically) ;  however, there are differences:
\begin{itemize}
\item polymorphic, higher-order API
\item more generic (cores/network)
\item does not focus on association lists
\item we don't have data locality (future work)
\end{itemize}

what is the target audience/applications: 
for instance, use if automatic provers on thousands of verification
conditions, on computing infrastructures which can be
\begin{itemize}
\item a single machine with multiple cores, possibly remote
\item several machines, small or large in number, over a network
\end{itemize}

\section{API}

main idea of the master function

four scenarios:
\begin{itemize}
\item cores
\item same binary
\item same version of ocaml
\item completely different programs
\end{itemize}

\section{Derived API}

map, map\_fold, etc.

generic programming using a functor, which is then applied in each of
the 4 cases

\section{Implementation Details}

cores?

network: protocol, fault tolerance, etc.

\section{Experiments}

n-queens, SMT solvers, etc.

\section{Future Work}

data locality: scheduling supplied by the user

% \appendix
% \section{Appendix Title}

% This is the text of the appendix, if you need one.

\acks

Acknowledgments, if needed.

% We recommend abbrvnat bibliography style.

\nocite{*}
\bibliographystyle{abbrvnat}
\bibliography{./biblio}

\end{document}
